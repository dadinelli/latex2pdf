\section{Requisiti}
In questa sezione verranno elencati tutti i requisiti specificati nel capitolato, organizzandoli in categorie distinte per una maggiore chiarezza. Ogni requisito è identificato in modo univoco da un codice, strutturato secondo un formato predefinito, che ne facilita la consultazione e il riferimento all'interno della documentazione con il seguente formato: 
\begin{center}
\textbf{R-[numero]-[tipo]-[priorità]}
\end{center}
dove:
\begin{itemize}
    \item \textbf{Numero:} un numero progressivo a tre cifre che parte dal numero 001.
    \item \textbf{Tipo:} la tipologia di requisito che può essere tra le seguenti:
    \begin{itemize}[label=-]
        \item \textbf{F (funzionale):} Una funzione di sistema descrive il modo in cui un sistema utilizza determinati ingressi per generare specifiche uscite, seguendo una logica o una regola prestabilita che definisce il suo comportamento.
        \item \textbf{Q (qualità):} definisce le caratteristiche di qualità del prodotto software.
        \item \textbf{V (vincolo):} specifica i limiti e le restrizioni imposte dal capitolato, che il prodotto software deve rispettare.
\end{itemize}
    \item \textbf{Priorità:} la priorità viene assegnata al requisito con un numero da 1 a 3:
    \begin{enumerate}
        \item Requisito obbligatorio che deve essere soddisfatto per la realizzazione del prodotto software.  
        \item Requisito desiderabile il cui soddisfacimento è apprezzato dal committente.
        \item Requisito facoltativo, la cui realizzazione è totalmente a discrezione del team in base all'andamento del progetto.
    \end{enumerate}
\end{itemize}
In alcuni casi sarà specificato nella colonna delle fonti se il requisito è stato esplicitamente indicato nel Capitolato oppure se è stato dedotto implicitamente da altri requisiti obbligatori. In quest’ultimo caso, si farà riferimento a un requisito interno.
\subsection{Registro di Funzionalità}
\input{contents/requisiti/requisiti_di_funzionalita}
\subsection{Registro di Qualità}
\input{contents/requisiti/requisiti_di_qualita}
\subsection{Requisiti di Vincolo}
\begin{table}[H]
    \begin{tabular}{|C{2.8cm}|C{9.5cm}|C{2.3cm}|}
        \hline
         \textbf{ID requisito} &  
         \textbf{Descrizione} &  
         \textbf{Fonti}  \\
          \hline
          R-XXX-X-X & Lorem ipsum dolor sit amet, consectetur adipiscing elit, sed do eiusmod tempor incididunt ut labore et dolore magna aliqua. & Capitolato \\
          \hline 
          R-XXX-X-X & Lorem ipsum dolor sit amet, consectetur adipiscing elit, sed do eiusmod tempor incididunt ut labore et dolore magna aliqua. & Interno \\
          \hline
    \end{tabular}

\end{table}